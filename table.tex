BACKUP SIN VATER:



\begin{table}[ht]
\centering
\begin{tabular}{lrrrrrrrrrr|r}
\toprule

Subset & DTC & LR & RFC & KNC & SVC & MLPC & XGBC & BC & ABC & GBC &Best \\
\midrule
50\% Data - Subset 1  & 0.970 & 0.538 & 0.944 & 0.969 & 0.476 & 0.969 & 0.967 & 0.966 & 0.970 & 0.970 & DCT \\
50\% Data - Subset 2  & 0.978 & 0.976 & 0.969 & 0.976 & 0.968 & 0.978 & 0.977 & 0.974 & 0.978 & 0.978 & ABC \\
50\% Data - Subset 3  & 0.994 & 0.991 & 0.996 & 0.993 & 0.995 & 0.995 & 0.996 & 0.995 & 0.995 & 0.996 & XGBC \\
50\% Data - Subset 4  & 0.995 & 0.993 & 0.996 & 0.989 & 0.995 & 0.996 & 0.997 & 0.996 & 0.996 & 0.997 & XGBC \\

100\% Data - Subset 1 & 0.970 & 0.539 & 0.944 & 0.969 & 0.741 & 0.969 & 0.967 & 0.966 & 0.970 & 0.970 & DCT \\
100\% Data - Subset 2 & 0.978 & 0.976 & 0.973 & 0.977 & 0.936 & 0.978 & 0.977 & 0.976 & 0.978 & 0.978 & ABC \\
100\% Data - Subset 3 & 0.995 & 0.991 & 0.996 & 0.994 & 0.994 & 0.995 & 0.996 & 0.996 & 0.995 & 0.996 & XGBC \\
100\% Data - Subset 4 & 0.995 & 0.993 & 0.997 & 0.990 & 0.995 & 0.997 & 0.997 & 0.997 & 0.996 & 0.997 & XGBC \\
\bottomrule
\end{tabular}
\caption{Accuracy scores of the best model parameters across different subsets and data configurations.}
\label{tab:accuracy_scores}
\end{table}





\begin{table}[]
\centering
\caption{Classification of the Attributes}\label{table:column-types}
\tiny
\begin{tabular}{llll}
\hline
\textbf{No} & \textbf{Column} & \textbf{Type} & \multicolumn{1}{c}{\textbf{Reasoning}} \\
\hline
1           & Unnamed: 0.1         & Nominal                 & Unique identifiers without inherent order. \\
2           & Unnamed: 0           & Nominal                 & Unique identifiers without inherent order. \\
3           & RA\_ICRS             & Interval                & ordered values with consistent differences; they lack an absolute zero, making ratios meaningless. \\
4           & DE\_ICRS             & Interval                & ordered values with consistent differences; they lack an absolute zero, making ratios meaningless; includes negative values, which precludes it from being ratio data. \\
5           & Source               & Nominal                 & Unique identifiers without inherent order. \\
6           & Plx                  & Ratio                   & quantifies celestial distance inversely in milliarcseconds, has a true zero (infinite distance), only positive values (distance can't be negative), and supports all arithmetic operations, allowing for direct distance comparison. \\
7           & PM                   & Ratio                   & measures motion with a true zero (no movement), only has positive values (motion magnitude), and arithmetic operations are valid, making distance comparison straightforward. \\
8           & pmRA                 & Ratio                & measure rate of change with true zero (no movement); negative values indicate direction, not absence of quantity; all arithmetic operations are meaningful, preserving the properties of ratio scales. \\
9           & pmDE                 & Ratio                & measure rate of change with true zero (no movement); negative values indicate direction, not absence of quantity; all arithmetic operations are meaningful, preserving the properties of ratio scales. \\
10          & Gmag                 & Interval                   & measures of apparent magnitude in different bands of light; on a logarithmic scale with an arbitrary zero point; differences, not ratios, are meaningful due to this scale. \\
11          & e\_Gmag              & Ratio                   & reflect uncertainty not quantity; they're tied to an interval scale (magnitude); and error ratios aren't meaningful, as they don't represent a doubled or halved error. \\
12          & BPmag                & Interval                   & measures of apparent magnitude in different bands of light; on a logarithmic scale with an arbitrary zero point; differences, not ratios, are meaningful due to this scale. \\
13          & e\_BPmag             & Ratio                   & reflect uncertainty not quantity; they're tied to an interval scale (magnitude); and error ratios aren't meaningful, as they don't represent a doubled or halved error. \\
14          & RPmag                & Interval                   & measures of apparent magnitude in different bands of light; on a logarithmic scale with an arbitrary zero point; differences, not ratios, are meaningful due to this scale. \\
15          & e\_RPmag             & Ratio                   &reflect uncertainty not quantity; they're tied to an interval scale (magnitude); and error ratios aren't meaningful, as they don't represent a doubled or halved error. \\
16          & GRVSmag              & Interval                   & measures of apparent magnitude in different bands of light; on a logarithmic scale with an arbitrary zero point; differences, not ratios, are meaningful due to this scale. \\
17          & e\_GRVSmag           & Ratio                   & reflect uncertainty not quantity; they're tied to an interval scale (magnitude); and error ratios aren't meaningful, as they don't represent a doubled or halved error. \\
18          & BP-RP                & Interval                & are interval data because they are differences of magnitudes; they lack a true zero and ratios are not meaningful; and they are based on a logarithmic scale, which is typical of interval data \\
19          & BP-G                 & Interval                &are interval data because they are differences of magnitudes; they lack a true zero and ratios are not meaningful; and they are based on a logarithmic scale, which is typical of interval data \\
20          & G-RP                 & Interval                & are interval data because they are differences of magnitudes; they lack a true zero and ratios are not meaningful; and they are based on a logarithmic scale, which is typical of interval data \\
21          & pscol                & Ratio                   & true zero point indicating no pseudocolor; it assumes only positive values; and arithmetic operations including ratios are meaningful for comparison. \\
22          & Teff                 & Ratio                   & measured in Kelvins, which has an absolute zero; and all arithmetic operations including ratios are valid, allowing for comparison of temperatures. \\
23          & Dist                 & Ratio                   & represents distance in parsecs with an absolute zero (no distance); and it supports all arithmetic operations, allowing for direct comparison of distances. \\
24          & Rad                  & Ratio                   & measured in solar radii, with a true zero (no size); and supports arithmetic operations, allowing for comparison of celestial sizes. \\
25          & Lum-Flame            & Ratio                   & a measure of brightness in solar luminosities, with a true zero (no luminosity); and arithmetic operations are valid, enabling comparative analysis of brightness. \\
26          & Mass-Flame           & Ratio                   & quantifies mass in units of solar mass, with an absolute zero (no mass); and it permits all arithmetic operations, facilitating comparisons of mass. \\
27          & Age-Flame            & Ratio                   & measures time in gigayears, with a true zero point (the beginning of the object's existence); and arithmetic operations, including ratios, are meaningful, allowing for age comparisons. \\
28          & z-Flame              & Ratio                   & it expresses redshift velocity in km\/s, with a true zero (no redshift); and allows for all arithmetic operations, enabling quantitative comparison of cosmic expansion effects. \\
29          & SpType-ELS           & Nominal                 & Attribute defining two distinct classes A and B. \\
\hline
\end{tabular}
\end{table}



\col{Unnamed: 0.1}, \col{Unnamed: 0}, \col{Source}: Unique identifiers without inherent order. -> nominal

\col{RA\_ICRS}, \col{DE\_ICRS}: Ordered values with consistent differences; they lack an absolute zero, making ratios meaningless. The latter also includes negative values, which precludes it from being ratio data. -> interval

\col{Plx}: Quantifies celestial distance inversely in milliarcseconds, has a true zero (infinite distance), only positive values (distance can't be negative), and supports all arithmetic operations, allowing for direct distance comparison. -> ratio

\col{PM}: Measures motion with a true zero (no movement), only has positive values (motion magnitude), and arithmetic operations are valid, making distance comparison straightforward. -> ratio

\col{pmRA}, \col{pmDE}: Measure rate of change with true zero (no movement); negative values indicate direction, not absence of quantity; all arithmetic operations are meaningful, preserving the properties of ratio scales. -> ratio

\col{Gmag}, \col{BPmag}, \col{RPmag}, \col{GRVSmag}: Measures of apparent magnitude in different bands of light; on a logarithmic scale with an arbitrary zero point; differences, not ratios, are meaningful due to this scale. -> interval

\col{e\_Gmag}, \col{e\_BPmag}, \col{e\_RPmag}, \col{e\_GRVSmag}: Reflect uncertainty not quantity; they're tied to an interval scale (magnitude); and error ratios aren't meaningful, as they don't represent a doubled or halved error. -> ratio

\col{BP-RP}, \col{BP-G}, \col{G-RP}: Are interval data because they are differences of magnitudes; they lack a true zero and ratios are not meaningful; and they are based on a logarithmic scale, which is typical of interval data. -> interval

\col{pscol}: True zero point indicating no pseudocolor; it assumes only positive values; and arithmetic operations including ratios are meaningful for comparison. -> ratio

\col{Teff}: Measured in Kelvins, which has an absolute zero; and all arithmetic operations including ratios are valid, allowing for comparison of temperatures. -> ratio

\col{Dist}: Represents distance in parsecs with an absolute zero (no distance); and it supports all arithmetic operations, allowing for direct comparison of distances. -> ratio

\col{Rad}: Measured in solar radii, with a true zero (no size); and supports arithmetic operations, allowing for comparison of celestial sizes. -> ratio

\col{Lum-Flame}: A measure of brightness in solar luminosities, with a true zero (no luminosity); and arithmetic operations are valid, enabling comparative analysis of brightness. -> ratio

\col{Mass-Flame}: Quantifies mass in units of solar mass, with an absolute zero (no mass); and it permits all arithmetic operations, facilitating comparisons of mass. -> ratio

\col{Age-Flame}: Measures time in gigayears, with a true zero point (the beginning of the object's existence); and arithmetic operations, including ratios, are meaningful, allowing for age comparisons. -> ratio

\col{z-Flame}: It expresses redshift velocity in km/s, with a true zero (no redshift); and allows for all arithmetic operations, enabling quantitative comparison of cosmic expansion effects. -> ratio

\col{SpType-ELS}: Attribute defining two distinct classes A and B. -> nominal



\begin{table}[]
\centering
\caption{Classification of the Attributes}\label{table:column-types}
\scriptsize
\begin{tabular}{lll}
\hline
\textbf{No} & \textbf{Column} & \textbf{Type} \\
\hline
1           & Unnamed: 0.1         & Nominal                 \\
2           & Unnamed: 0           & Nominal                 \\
3           & RA\_ICRS             & Interval                \\
4           & DE\_ICRS             & Interval                \\
5           & Source               & Nominal                 \\
6           & Plx                  & Ratio                   \\
7           & PM                   & Ratio                   \\
8           & pmRA                 & Ratio                \\
9           & pmDE                 & Ratio                \\
10          & Gmag                 & Interval                   \\
11          & e\_Gmag              & Ratio                   \\
12          & BPmag                & Interval                   \\
13          & e\_BPmag             & Ratio                   \\
14          & RPmag                & Interval                   \\
15          & e\_RPmag             & Ratio                   \\
16          & GRVSmag              & Interval                   \\
17          & e\_GRVSmag           & Ratio                   \\
18          & BP-RP                & Interval                \\
19          & BP-G                 & Interval                \\
20          & G-RP                 & Interval                \\
21          & pscol                & Ratio                   \\
22          & Teff                 & Ratio                   \\
23          & Dist                 & Ratio                   \\
24          & Rad                  & Ratio                   \\
25          & Lum-Flame            & Ratio                   \\
26          & Mass-Flame           & Ratio                   \\
27          & Age-Flame            & Ratio                   \\
28          & z-Flame              & Ratio                   \\
29          & SpType-ELS           & Nominal                 \\
\hline
\end{tabular}
\end{table}


\col{GRVSmag}: About 40\% of the data is missing, we cannot throw away all data. Imputing with he median is advised to maintain robustness. (median for Group A is about 11.4 and median for group B is about 10.8)
\col{e\_GRVSmag}: About 40\% of the data is missing, we cannot throw away all data. Imputing with he median is advised to maintain robustness. (median for both groups about 0.01)
\col{Age-Flame}: About 25\% of the data is missing, we cannot throw away all data. Imputing with he median is advised to maintain robustness. (median for Group A is about 0.75 and median for group B is about 0.3)


In a folliwng step, the not yet standardized columns were normalisted. For that the follwoing python code was applied to identify relevant standardistaiton techniques. The corresponding hyperparemters (e.g., \texttt{SKEWNESS\_THRESHOLD} and \texttt{OUTLIER\_COUNT\_THRESHOLD}) were not speficically tuned, but instead a best-effort appraoch was taken. After that, the reamining colums were preprocessed using the Python code depicted in Listing~\ref{lst:suggest_scaler}.





Based on the description of attributes from the Gaia catalogue data, here are some interesting findings or studies that could be suggested:

Galactic Mapping: With the RA_ICRS and DE_ICRS, along with distance (Dist), you can create a three-dimensional map of a section of the Milky Way galaxy. This can help in understanding the structure of our galaxy.

Stellar Kinematics: Using proper motion (PM, pmRA, pmDE) and redshift (z-Flame), researchers can study the movement of stars within the galaxy. This can lead to insights into the dynamics of stellar populations and the gravitational influence of dark matter.

Star Formation Histories: The Age-Flame attribute can be used to investigate the distribution of stellar ages within various regions of the galaxy, potentially revealing star formation rates and histories.

Stellar Evolution: By examining Lum-Flame, Teff, Mass-Flame, and Rad, astronomers can study how stars evolve over time and how they end their life cycles—whether they'll become white dwarfs, neutron stars, or black holes.

Stellar Populations: The SpType-ELS can be used to categorize stars into different spectral types, which can be correlated with other attributes like Teff and Lum-Flame to classify and study different populations of stars.

Search for Exoplanets: The Gaia catalogue's precise measurements of stellar positions and movements can be used to detect slight wobbles in a star’s position, which might indicate the presence of orbiting exoplanets.

Distance Ladder Calibration: With precise parallax measurements (Plx), Gaia data can be used to calibrate the cosmic distance ladder, which is fundamental for measuring distances in the universe.

Astrophysical Phenomena: Correlating color indices (BP-RP, BP-G, G-RP) with apparent magnitudes (Gmag, BPmag, RPmag) can help identify astrophysical phenomena such as interstellar extinction and the intrinsic luminosity of stars.

Metallicity Studies: Pseudocolor (pscol) is related to the amount of metal in a star's spectrum. Analyzing this in conjunction with spectral class (SpType-ELS) could help in understanding the metallicity distribution in the galaxy.

Gravitational Lensing Events: The precise position and movement data may help in identifying stars that act as gravitational lenses, leading to the discovery of more distant objects through the lensing effect.

Dark Matter Mapping: By studying the movements and velocities of stars (using PM, pmRA, pmDE, z-Flame), scientists can infer the distribution of dark matter, which affects the motion of stars through its gravitational effects.




\subsection{Conclusion}

\subsubsection{}*{Correlations in Magnitudes and Color Indices}
\begin{itemize}
  \item Magnitudes across different bands (\textit{GRVSmag, RPmag, BPmag, Gmag}) are correlated, indicating a consistency in the assessment of a star's luminosity across these bands.
  \item Color indices (\textit{BP-RP, BP-G, G-RP}) show strong inter-correlations. They are derived from the magnitudes and offer insights into the star's temperature, with the color becoming bluer as the temperature increases.
  \item A notable correlation exists between parallax (\textit{Plx}) and color indices, suggesting distance may affect perceived stellar properties, potentially due to interstellar reddening.
\end{itemize}

\subsubsection*{Mass-Luminosity Relation and Measurement Errors}

\begin{itemize}
  \item A clear correlation is observed between mass (\textit{Mass-Flame}) and luminosity (\textit{Lum-Flame}), supporting the mass-luminosity relationship, where more massive stars are generally more luminous and hotter (\textit{Teff}).
  \item Errors in magnitude measurements (\textit{e\_BPmag, e\_RPmag}) are likely correlated due to similar sources of uncertainty in the observation process.
\end{itemize}

\subsubsection*{Distribution Characteristics and Missing Data}

\begin{itemize}
  \item Missing data is most prevalent for \textit{GRVSmag}, suggesting limitations in instrument sensitivity or data processing approaches.
  \item The distribution of \textit{GRVSmag} shows an unusual drop, indicating potential areas for further investigation.
\end{itemize}

\subsubsection*{Data Distributions and Group Comparisons}

\begin{itemize}
  \item Attributes such as \textit{pmRA, pmDE, e\_Gmag, e\_BPmag, e\_RPmag, e\_GRVSmag, BP-RP, BP-G, G-RP, Rad, Lum-Flame, z-Flame} exhibit similar distributions, suggesting these properties may behave consistently across different groups.
  \item In contrast, \textit{RA\_ICRS, DE\_ICRS, PLX, PM, Gmag, BPmag, RPmag, GRVSmag} show divergent distributions, which may reflect intrinsic differences in star populations or observational biases.
  \item Some attributes like \textit{Teff, Dist, Mass-Flame, Age-Flame} stand out for their ability to differentiate between groups, indicating their importance in stellar classification.
\end{itemize}






Although some really interesting patterns occur with any combination with \col{Source}, \col{Unnamed} and \col{Unnamed} it does not make sense to investigate those. This is due to the fact that does are arbitrary indexes and do not provide more information to the data. 

We identified highly correlated groups, which can be explained as follows. 

\begin{itemize}
  \item The magnitudes \col{GRVSmag}, \col{RPmag}, \col{BPmag}, and \col{Gmag} are all brightness measures, hence they are naturally correlated as they reflect a star's overall luminosity.
  \item Color indices \col{BP-RP}, \col{BP-G}, and \col{G-RP} derive from the aforementioned magnitudes; their strong inter-correlations are due to their shared basis in these magnitudes.
  \item The correlation between color indices and \col{Plx} suggests that parallax, which is indicative of distance, might influence perceived color properties.
  \item The \col{Mass-Flame}'s correlation with both \col{Teff} and \col{Lum-Flame} aligns with the mass-luminosity relationship and the concept that more massive stars are typically hotter.
  \item Errors in magnitude measurements, \col{e\_BPmag} and \col{e\_RPmag}, are correlated likely due to similar sources of observational uncertainty: Using the same instruments will most likely introduce the same errors.
\end{itemize}

The missing data analysis reveals:
\begin{itemize}
  \item Most of the data is missing for \col{GRVSmag}.
  \item The overall amount of outliers is relatively high, reaching from 0 to 63 (2.1\% of the total data).
  \item Column \col{GRVSmag} has a very odd distribution (sudden drop at 12). It seems like missing data that would be useful to investigate more.
\end{itemize}

The boxplots, boxenplots, and violin plots reveal:
\begin{itemize}
  \item The following attributes have a similar distribution, subsequently also similar values in mean, median etc with respect to the different groups \texttt{A} and \texttt{B}: \col{pmRA}, \col{pmDE}, \col{e\_Gmag}, \col{e\_BPmag}, \col{e\_RPmag}, \col{e\_GRVSmag}, \col{BP-RP}, \col{BP-G}, \col{G-RP}, \col{Rad}, \col{Lum-Flame}, \col{z-Flame}.
  \item The following attributes have a very different distribution, subsequently also different values in mean, median etc with respect to the different groups \col{A} and \col{B}: \col{RA\_ICRS}, \col{DE\_ICRS}, \col{PLX}, \col{PM}, \col{Gmag}, \col{BPmag}, \col{RPmag}, \col{GRVSmag}.
  \item Noteworthy to mention are the very good attributes separable: \col{Teff}, \col{Dist}, \col{Mass-Flame}, \col{Age-Flame}.
\end{itemize}



\subsection*{Patterns and Correlations}

We observe that combinations involving \col{Source} or columns with no specific names (referred to as \col{Unnamed}) show patterns, but these are not explored further due to their nature as arbitrary indices without additional informative value.

\subsection*{Correlated Groups}

\begin{itemize}
  \item The magnitudes \col{GRVSmag}, \col{RPmag}, \col{BPmag}, and \col{Gmag} reflect a star's overall luminosity and are naturally correlated.
  \item Color indices \col{BP-RP}, \col{BP-G}, and \col{G-RP} derive from magnitudes and exhibit strong inter-correlations.
  \item A correlation between color indices and parallax (\col{Plx}) suggests distance may influence perceived color properties.
  \item Mass (\col{Mass-Flame}) correlates with temperature (\col{Teff}) and luminosity (\col{Lum-Flame}), consistent with the mass-luminosity relationship.
  \item Errors in magnitude measurements (\col{e\_BPmag} and \col{e\_RPmag}) are correlated, indicating similar sources of observational uncertainty.
\end{itemize}

\subsection*{Missing Data and Distribution Characteristics}

\begin{itemize}
  \item A significant proportion of data is missing for \col{GRVSmag}, necessitating further investigation.
  \item Anomalies such as a distribution "drop" at the value of 12 for \col{GRVSmag} are observed, suggesting possible areas for future research.
\end{itemize}

\subsection*{Comparative Analysis of Distributions}

Boxplots and violin plots reveal:

\begin{itemize}
  \item Attributes such as \col{pmRA}, \col{pmDE}, \col{e\_Gmag}, \col{e\_BPmag}, \col{e\_RPmag}, \col{e\_GRVSmag}, \col{BP-RP}, \col{BP-G}, \col{G-RP}, \col{Rad}, \col{Lum-Flame}, and \col{z-Flame} show similar distributions across groups \texttt{A} and \texttt{B}.
  \item Distributions and central tendency measures (mean, median) for \col{RA\_ICRS}, \col{DE\_ICRS}, \col{PLX}, \col{PM}, \col{Gmag}, \col{BPmag}, \col{RPmag}, and \col{GRVSmag} vary between the groups.
  \item Attributes such as \col{Teff}, \col{Dist}, \col{Mass-Flame}, and \col{Age-Flame} provide clear separation between groups, highlighting their discriminative power.
\end{itemize}



\subsection{Interesting Clusters for further examination}
\label{subsection:interesting_clusters_for_further_examination}

Using the scatter plots, we can reveal the following interesting groups to be examined further
\begin{itemize}
    \item \gra has a \col{Teff} of 7000 - 10000 and a \col{Age-Flame} from 0.25 up to 1.5.
    \item \grb has a \col{Teff} of 10000 - 200000 and a \col{Age-Flame} from 0.25 up to 0.5.

    \item \gra has a \col{Teff} of 7000 - 10000 and a \col{Mass-Flame} from 1.3 to 3.
    \item \grb has a \col{Teff} of 10000 - 200000 and a \col{Mass-Flame} from 2 to 7.

    \item \gra has a \col{Teff} of 7000 - 10000 and a \col{RPmag} from 6 to 14.
    \item \grb has a \col{Teff} of 10000 - 200000 and a \col{RPmag} from 9 to 17.

    \item \gra has a \col{Teff} of 7000 - 10000 and a \col{Rad} from 1 to 6.
    \item \grb has a \col{Teff} of 10000 - 200000 and a \col{Rad} from 1 to 6.
\end{itemize}

Cluster anaylsis with PCA, UMAP and t-SNE provides evidence that PCA did not really work out, whereas the other two nonlinear methods were able to idnetify clusters. Not only could 2 distinct groups beeen formed, they also match the classes quite well. This suggests that the data contains some non-linear nature. Refering to the hierachichal clusteirng, it would be ineresting if there are any subgroups within \gra and \grb present.



\subsection{Description of Data Mining Problem}
The objective of this project is to develop a classifier to predict the "SpType-ELS" attribute, which represents the estimated spectral class by Gaia. This attribute is categorical with possible values "A" and "B". The goal is to leverage various state-of-the-art classification algorithms to accurately classify spectral types based on the provided dataset.

\subsection{Task Details}
As a data scientist, you will build and evaluate multiple classifiers using techniques covered in lectures. The tasks include:
\begin{itemize}
    \item Preprocessing the data.
    \item Building classifiers using KNIME or Python.
    \item Evaluating and selecting the best model based on performance metrics.
    \item Explaining the chosen model and its parameters.
\end{itemize}

\subsection{Datasets}
Three datasets are provided:
\begin{itemize}
    \item \textbf{Training Dataset:} For training and optimizing the model.
    \item \textbf{Unknown Dataset:} For final model assessment.
    \item \textbf{Sample Submission:} Shows the correct format for submitting predictions to Kaggle.
\end{itemize}

\subsection{Classification Task}
The primary task is to build a classifier for the "SpType-ELS" attribute. Effective preprocessing, feature engineering, and thorough evaluation are crucial for identifying the best-performing model.

\subsection{Submission}
Submit the predictions as a .csv file on Kaggle, ensuring the format aligns with the sample submission.
